% (C) 2016 Jean Nassar. Some Rights Reserved
% Except where otherwise noted, this work is licensed under the Creative Commons Attribution-ShareAlike License, version 4
% Title page
\renewcommand{\maketitlehooka}{
  \begin{center}
    \noindent

    \begin{figure}[h]
      \centering
      \includegraphics[width=0.3\textwidth]{kyoto_u_logo}
    \end{figure}
    \textsc{\LARGE Kyoto University}
    
    \textsc{\Large Graduate School of Engineering}

    \textsc{\large Mechanical Engineering Science}
  \end{center}
  \vfill\vfill
}
\renewcommand{\maketitlehookb}{
  \vfill\vfill
  \begin{center}
  {\large A thesis presented for the degree of}

  \textsc{\large Master of Science}
  \end{center}
  \vfill\vfill
}
\renewcommand{\maketitlehookc}{
  %\vfill
}
%\renewcommand{\maketitlehookd}{
  %\begin{center}
    %\begin{figure}[h]
      %\centering
      %\includegraphics{by-sa-big}
    %\end{figure}
  %\end{center}
%}

\title{\mytitle}
% \author{Jean \textsc{Nassar}\thanks{Supervisor: Dr.~Fumitoshi Matsuno}}
\author{Jean \textsc{Nassar} \\\emph{Author} \and Dr.~Fumitoshi \textsc{Matsuno}\\\emph{Supervisor}}
\date{\today}

\begin{titlingpage}
\maketitle
\thispagestyle{empty}

% Copyright page
\clearpage
\null\vfill
\noindent \emph{\mytitle}

\noindent \emph{\mytitlejp}
\vspace{1em}

\noindent All the source code for this project is accessible at the author's public Github profile (\url{https://github.com/masasin}).
The source code for the \acrshort{spirit} system software, the survey, the experimental data analysis, and the thesis itself are available in the \href{https://github.com/masasin/spirit}{\textsf{spirit}}, \href{https://github.com/masasin/spirit_survey}{\textsf{\detokenize{spirit_survey}}}, \href{https://github.com/masasin/spirit_analysis}{\textsf{\detokenize{spirit_analysis}}}, and \href{https://github.com/masasin/spirit_thesis}{\textsf{\detokenize{spirit_thesis}}} repositories respectively.
\vspace{1em}

\noindent This document was typeset using the \XeTeX\ typesetting system created by the Non-Roman Script Initiative and the \textsf{memoir} class created by Peter Wilson.
The text is set in \mytextsize\ in Linux~Libertine~O.
The other fonts used include \textsf{Linux~Biolinum~O}, \texttt{DejaVu~Sans~Code}, and IPA~Mincho~(明朝).
\vspace{1em}

\noindent \includegraphics{by-sa-big}

\noindent
% \includegraphics{by-sa}
Copyright \copyright 2017 by Jean Nassar.

\noindent This thesis is licensed under the Creative Commons Attribution-ShareAlike 4.0 International License.
To view a copy of this license, visit:\\
\url{http://creativecommons.org/licenses/by-sa/4.0/}

\noindent The \acrshort{spirit} software is licensed under the Berkeley Software Distribution 3-Clause License, except where otherwise noted.

\noindent The code used for analysis is licensed under the MIT License.
\end{titlingpage}


% Dedication
%\clearpage
\thispagestyle{empty}
\setlength{\epigraphwidth}{0.7\textwidth}
\null\vfill
\epigraph{
  because [\ldots] if you don't know where you are, then you don't know where you're going.
  And if you don't know where you're going, you're probably going wrong.}
  {\emph{I Shall Wear Midnight}\\
   Terry \textsc{Pratchett}}
\vfill\vfill\null
  %\null\vspace{\stretch{1}}
    %\noindent
    %because [\ldots] if you don't know where you are, then you don't know where you're going.
    %And if you don't know where you're going, you're probably going wrong.

    %\begin{flushright}
      %\emph{I Shall Wear Midnight}\\
      %\textsc{Terry Pratchett}
    %\end{flushright}
  %\vspace{\stretch{2}}\null

% Abstract
\clearpage
\thispagestyle{empty}
\null\vfill
\renewcommand{\abstractname}{Summary}
\begin{abstract}
  \gls{spirit} is a system which provides an operator with a third-person perspective of a robot with respect to its environment.
  The view is generated by overlaying a \gls{3d} model of the robot onto a previously-acquired image, such that the model would be in the \gls{fov} of the image.

  This research focuses on developing a \gls{spirit}-based user interface for aerial robots.
  The proposed method combines \gls{fov} information with state estimation, and selects a suitable image to use as a background.
  This method can work at low bandwidths and framerates, and only requires a single camera.

  Using the proposed interface, the operator's situational awareness can be significantly increased.
  Experiments have shown that the accuracy increased, and errors in the \sym{posx} direction significantly decreased.
  However, there was a slight increase in time needed to complete the task, and the total length of the path.
\end{abstract}
\vfill\vfill\null


% Acknowledgements
\clearpage
\thispagestyle{empty}
\null\vfill
\renewcommand{\abstractname}{Acknowledgements}
\begin{abstract}
  I would like to thank everyone who supported and assisted me, directly and indirectly.
\end{abstract}
\vfill\vfill\null


% Table of contents etc
\clearpage
\tableofcontents
\clearpage
\listoffigures
\listoftables

% Preface
\glsresetall  % Reset acronym usage.
