% (C) 2016 Jean Nassar. Some Rights Reserved
% Except where otherwise noted, this work is licensed under the Creative Commons Attribution-ShareAlike License, version 4
% Results
\chapter{Results}
\label{ch:results}
%\epigraphhead{\epigraph{
%Usability answers the question, ``Can the user accomplish their goal?''}{\textsc{Joyce Lee}}}
The numerical results of the experiments are shown in \tref{tab:mean_sd}, and are separated into their own experiment types.
Their \glspl{hpd} are not shown in the table, but they are mentioned when needed.
The difference of means and the effect sizes (Hedges's \sym{effect}) are shown in \tref{tab:diff_means}, along with the 95\% \gls{ci} of each, in parentheses.
As a general rule of thumb, an effect can be considered statistically significant at a desired level if the \gls{ci} does not contain zero.
Values of 0.2, 0.5, and 0.8 can be considered small, medium, and large effects respectively.

  \section{Path length}
  \fref{fig:paths_overview} shows the paths flown by the operators, as well as the locations at which they arrived.
  The target is the coloured rectangle, while the dashed line represents the area within which one part of the drone would be over the target.

  \begin{figure}[h]
    \centering
    \input{img/plots/paths_overview.pgf}
    \caption[Paths overview]{Overview of the paths flown by the operators and their arrival locations.}
    \label{fig:paths_overview}
  \end{figure}

  All \gls{spirit} subjects confirmed that they were able to see the target, as well as their relative position in \sym{posx}, but many had difficulty estimating their position in \sym{posy}.
  This can be seen by the amount of paths which overfly the target area completely compared to the onboard view.
  It seems notable that onboard pilots erred on the side of caution and undershot their approach.
  This may be due to the fact that the target disappears earlier when using the onboard view, and, with no depth perception, the user cannot rely on environmental cues.

  The tendency of the drone to yaw to the right can clearly be seen, where users initially start out by moving to the right before correcting their path.
  In addition, \gls{spirit} users tend to take a more consistent route to their destination, and stay in a more narrow zone.
  This might indicate that they are more comfortable with the interface.

  The path length is shown in \fref{fig:movement}.
  An ideal run with zero wasted motion would have a total path length of 6.0\,m.
  However, the mean path length for runs using the onboard view was $10.859 \pm 2.703$\,m, while that for \gls{spirit} runs was $11.913 \pm 2.683$\,m.
  That is, \gls{spirit} users flew 1.03\,m longer than their onboard counterparts.
  It appears that there is a small effect size (\sym{effect}=$-0.391$), but the \gls{ci} is only 86.1\%.


  \begin{figure}[h]
    \centering
    \input{img/plots/movement.pgf}
    \caption[Path lengths]{Path length, including total movement in \sym{posx} and \sym{posy}.}
    \label{fig:movement}
  \end{figure}

  There was almost no difference ($-0.105$\,m) in total motion in the \sym{posx} direction, but there was a small, nonsignificant correlation in \sym{posy} ($\Delta$\sym{mean}=$-0.223$\,m, \sym{effect}=0.359, \gls{ci}=83.6\%).

  From \fref{fig:movement_runs}, it appears that the first and third \gls{spirit} flights produced longer path lengths.
  In both these cases, the operator was using the system for the first time.
  Towards the end of that first flight, and continuing into their next attempt, the path lengths are comparable to those flown with the onboard view.
  This may be a statistical aberration, or it could be the effect of familiarity with the system.

  \begin{figure}[h]
    \centering
    \input{img/plots/movement_runs.pgf}
    \caption[Path lengths across runs]{The change in path length across runs. The movement in \sym{posx} and \sym{posy} was larger with \gls{spirit} on the first and third runs, but is similar in subsequent runs.}
    \label{fig:movement_runs}
  \end{figure}

  \section{Accuracy}
  \fref{fig:paths_detailed} shows the location of all the arrival points in each of the groups with respect to the target.

  \begin{figure}[h]
    \centering
    \input{img/plots/paths_detailed.pgf}
    \caption[Arrival overview]{Detail of the arrival points with respect to the target. Lighter points represent later runs.}
    \label{fig:paths_detailed}
  \end{figure}

  Out of the eighteen arrivals, none of the onboard ones are directly above the target, and only seven had a portion of the drone above the target.
  The distribution is very wide and not precise, and was slightly semicircular.
  Some students have commented that they were using a motion capture camera pole as a marker, and orienting themselves around that.

  By contrast, \gls{spirit} operators obtained a much more consistent result, with both higher accuracy and precision.
  Four out of eighteen were directly above the target, and a further eight had a portion of the drone above the target.
  This distribution is much more concentrated.

  \fref{fig:distance} shows the distance from the target at the time of arrival.
  As with the paths, people using the onboard view had a strong backward bias, with a mean of $-0.474$\,m.
  The 95\% \gls{hpd} was between $-0.747$ and $-0.209$\,m, which is completely outside of the target region.
  They also had a large standard deviation in \sym{posx} despite having a mean close to the centre (\sym{mean}=0.038\,m, \sym{std}=0.43\,m).


  \begin{figure}[h]
    \centering
    \input{img/plots/distance.pgf}
    \caption[Arrival distance]{Distance from target, including in \sym{posx} and \sym{posy}, at the time of arrival.}
    \label{fig:distance}
  \end{figure}

  Meanwhile, the \gls{spirit} view had a strong rightward bias, with a mean of 0.136\,m, and a 95\% \gls{hpd} of 0.033 to 0.237\,m.
  Nevertheless, the 95\% \gls{hpd} is completely above the target.
  On the other hand, while the mean of the \sym{posy} error ($-0.067$\,m) is above the target, its large standard deviation 0.366\,m means that the users can be off-target by up to about one target length (95\% \gls{hpd}: $-0.287$ to 0.141\,m).

  \glspl{rmse} for \sym{posx} and \sym{posy}, as well as the total distance to the target at the time of arrival, are shown in \fref{fig:rmse}.
  The total \gls{rmse} was almost identical to the total distance shown in \fref{fig:distance}.

  \begin{figure}[h]
    \centering
    \input{img/plots/rms.pgf}
    \caption[Arrival RMS Error]{\gls{rmse} in distance from target, including in \sym{posx} and \sym{posy} at the time of arrival.}
    \label{fig:rmse}
  \end{figure}

  While the onboard view had an error of $0.666\pm0.284$\,m, \gls{spirit} had an error of only $0.401\pm0.238$\,m.
  Broken down, the \sym{posx}- and \sym{posy}-\glspl{rmse} for the onboard view were $0.338\pm0.191$\,m and $0.52\pm0.293$\,m respectively.
  This compares to just $0.192\pm0.122$\,m and $0.338\pm0.222$\,m for \gls{spirit}, respectively.
  
  The difference is significant and the effect is large for the total distance ($\Delta$\sym{mean}=0.266\,m, \sym{effect}=1.053, \gls{ci}=98.1\%) and \gls{rmse}$_{\sym{posx}}$ ($\Delta$\sym{mean}=0.143\,m, \sym{effect}=0.924, \gls{ci}=97.8\%).
  They are nonsignifanct and medium, respectively, for \gls{rmse}$_{\sym{posy}}$ ($\Delta$\sym{mean}=0.182\,m, \sym{effect}=0.731, \gls{ci}=95.0\%).

  \fref{fig:rms_runs} shows the change in the \glspl{rmse} across runs.
  It is consistently lower with \gls{spirit}, apart for the second run, which is slightly higher than the onboard view.
  This may be an aberration.

  \begin{figure}[h]
    \centering
    \input{img/plots/rms_runs.pgf}
    \caption[Arrival RMS Error across runs]{Arrival \gls{rmse} change across runs.}
    \label{fig:rms_runs}
  \end{figure}

  \section{Duration}
  \fref{fig:duration_result} shows the duration of each type of experiment.
  \gls{spirit} seemed to have a longer duration, but with less spread.
  In fact, the onboard view has a mean duration of 39.041\,s, and a standard deviation of 18.817\,s, while \gls{spirit} has $44.081\pm15.02$\,s.

  This difference, though, is not signficant, and the effect size is small. For duration, $\Delta$\sym{mean}=-5.041\,s, \sym{effect}=0.304, \gls{ci}=81.5\%.
  
  \begin{figure}[h]
    \centering
    \begin{subfigure}[b]{0.45\textwidth}
      \input{img/plots/duration.pgf}
      \caption{Duration of each type of experiment.}
      \label{fig:duration_result}
    \end{subfigure}
    \hfill
    \begin{subfigure}[b]{0.45\textwidth}
      \input{img/plots/duration_runs.pgf}
      \caption{Change in duration across runs.}
      \label{fig:duration_runs}
    \end{subfigure}
    \caption[Duration]{The duration of the flight, from takeoff until the arrival button was pressed.}
    \label{fig:duration}
  \end{figure}

  Looking at \fref{fig:duration_runs}, the difference between the durations decreased in each run, and \gls{spirit} was faster by the fourth run.
  This could be indicative of ease of learning, since the improvements were being made faster than with the onboard version.
  Further experimentation is needed to verify this hypothesis.

  \section{Workload}
  \fref{fig:tlx} shows the result of the \gls{nasatlx} survey, both in aggregate and by component.
  Because the \gls{tlx} responses are subjective, and the scale itself is an ordinal rather than interval scale, no actionable information can be gleaned from this small a sample size.\cite{hart2006}
  Instead, general trends may be observed.

  The six components in \fref{fig:tlx_components} are, in order:

  \begin{itemize}
    \item \textbf{\acrshort{mental}:} \acrlong{mental}
    \item \textbf{\acrshort{physical}:} \acrlong{physical}
    \item \textbf{\acrshort{temporal}:} \acrlong{temporal}
    \item \textbf{\acrshort{performance}:} \acrlong{performance}
    \item \textbf{\acrshort{effort}:} \acrlong{effort}
    \item \textbf{\acrshort{frustration}:} \acrlong{frustration}
  \end{itemize}

  \noindent and the weighted score is \acrshort{tlxscore}.
  
  Most onboard pilots who had a high score for physical demand mentioned that it was due to the fact that they needed to move in short burts to keep from hitting their surroundings.
  Since the frame rate was so low, the drone would move a significant distance by the time a frame updated.
  This raised stress and caused some frustration.

  Inherent issues in the system, such as with the drone's tendency to drift right, or the lack of depth perception, also contributed to frustration and mental demand, but it had a similar effect when using either system.

  \begin{figure}[h]
    \centering
    \begin{subfigure}[b]{0.45\textwidth}
      \input{img/plots/tlx_results.pgf}
      \caption{\gls{nasatlx} aggregate results.}
      \label{fig:results}
    \end{subfigure}
    \hfill
    \begin{subfigure}[b]{0.45\textwidth}
      \input{img/plots/tlx_components.pgf}
      \caption{\gls{nasatlx} component analysis.}
      \label{fig:tlx_components}
    \end{subfigure}
    \caption[NASA-TLX results]{Results for the \gls{nasatlx} survey.}
    \label{fig:tlx}
  \end{figure}

  The collated data is in \tref{tab:mean_sd_tlx}, and analyzed in \tref{tab:diff_means_tlx}.
  A large, significant reduction of 35.74\% in the weighted \gls{tlx} score was seen.
  Analysis of the components show that the scores decreased across the board.
  There was a medium to high effect for physical (\sym{effect}=$-0.626$), temporal (\sym{effect}=$-0.666$), performance (\sym{effect}=$-0.699$) effort (\sym{effect}=$-0.718$), and overall score (\sym{effect}=$-0.978$).

  \section{Survey}
  \begin{figure}[h]
    \centering
    \input{img/plots/survey_overview.pgf}
    \caption[Survey results]{Results for the exit survey.}
    \label{fig:survey}
  \end{figure}

  \fref{fig:survey} shows the result of the exit survey.
  Again, scores increased for each category.

  The collated data is in \tref{tab:mean_sd_survey}, and analyzed in \tref{tab:diff_means_survey}.
  \gls{spirit} did not affect orientation awareness or control.
  However, it significantly increased awareness and control for both absolute and relative positioning.
  The aggregate score also significantly increased.

  The largest effect was on the abillity to stay above the target (\sym{effect}=2.511, $p$=0.00126).
  One strategy that \gls{spirit} users utilized was slewing to the side in order to get a perspective view of the location of the drone with respect to the target.
  By extrapolating the vertical edges of the target, they were able to increase their understanding of the situation.

